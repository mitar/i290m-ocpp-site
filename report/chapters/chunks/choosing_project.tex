\mysubsubsection{The Reasons for Project Choice}
Much has been written about motivations in open source software
contributions \cite{vonKrogh2012}. It is insightful to survey what are the main reasons 
for people when choosing their venue for participation -- particularly among
our class population, which is made up of people with a strong interest in peer production. 
The overwhelming majority ($67\%$) of respondents reported that the project goals -- i.e, the audience and functionality --  was the most important aspect. Technical matters ( including the state of the project and the
skills required ) and community norms trailed far behind, with 17\% and 14\%
respectively. One respondent chose {\bf Other}, commenting ``my own interests and skills I can get". Even this statement does not necessarily go against the fact her/his own interests where aligned with the audience and even more with skills expected to be acquired. The narrative elaborations reveal that motivations related to project goals are determined by alignment with personal values, with skill sets. Some interesting quotes from the blog posts illustrate further the spirit of joining a project :

\begin{quotation} 
``As a self-confessed idealist,  what really sold me about the {\it Hypothes.is} project was the their huge plans. I liked the big picture thinking."
\end{quotation}

\begin{quotation}
``A shared, peer-distributed datastore is its own reward."
\end{quotation}

\begin{quotation}
``I wasn't the most technically sound person so I knew that I wouldn't 
necessarily be contributing to the coding of the project. I wanted to be apart of a 
project that did good in the world because I believe the power of open source is 
underutilized in philanthropy."
\end{quotation}

\begin{quotation}
``To me, all open source projects have a benign motto. Therefore, my decision criteria mainly
 depended on the technical aspects of the project. I want to contribute in an area where I 
 can understand their system better and contribute to the best of my abilities."
 \end{quotation}

\begin{quotation}
``I am very interested in technology that facilitates spreading and creating knowledge, 
and have struggled first-hand with the current limitations in the process of consuming, 
sharing, and building upon scientific literature."
\end{quotation}

Of course, people elaborated on the other motivations as well:

\begin{quotation}
``Project seemed like a good technical challenge, from which I could improve my Javascript 
skills and general code organization."
\end{quotation}

\begin{quotation}
``I wanted a community that was open and friendly, and I found that to be more 
important than anything else."
\end{quotation}
